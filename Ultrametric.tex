\documentclass{llncs}
\usepackage{graphicx}
\usepackage{float}
\usepackage{amssymb,amsmath}

\begin{document}

\title{On State Complexity of Ultrametric Finite Automata}


\author{
Kaspars Balodis,
Anda Beri\c na,
Krist\= \i ne C\= \i pola,
J\= anis Iraids,
K\= arlis J\= eri\c n\v s,
J\= anis Kal\= ejs,
Rihards Kri\v slauks,
K\= arlis Luksti\c n\v s,
Nat\= alija Somova,
Irina \v S\v cegu\c lnaja,
Anna Vanaga,
R\= usi\c n\v s Freivalds}
\institute{Institute of Mathematics and Computer Science,
 University of Latvia,\\ Rai\c na bulv\= aris 29, Riga, LV-1459, Latvia
}



\maketitle

\begin{abstract}  
We introduce a notion of ultrametric finite automata using $p$-adic numbers to describe random branching of the process of computation. These automata have properties similar to the properties of probabilistic automata but complexity of probabilistic automata and complexity of ultrametric automata can differ very much.
TODO
\end{abstract} 



\section{Introduction} 
TODO: kaut ko sarakstiit / sakopeet no tiem 2 rakstiem 

\section{$p$-adic numbers}
TODO: kaut ko sarakstiit / sakopeet no tiem 2 rakstiem 

\section{Results}

TODO: veel teoreemas

\begin{theorem}
For every $n>0$ there exists a language $L_n$ such that every deterministic finite automaton recognizing $L_n$ needs at least $2^n$ states, but there is an Ultrametric automaton recognizing $L_n$ with $2 n$ states.
\end{theorem}
\begin{proof}
TODO
\qed
\end{proof}

TODO: veel teoreemas

\end{document}
