\documentclass{llncs}
\usepackage{graphicx}
\usepackage{float}
\usepackage{amssymb,amsmath}

\begin{document}

\title{On State Complexity of Ultrametric Finite Automata}


\author{
Kaspars Balodis,
Anda Beri\c na,
Krist\= \i ne C\= \i pola,
J\= anis Iraids,
K\= arlis J\= eri\c n\v s,
J\= anis Kal\= ejs,
Rihards Kri\v slauks,
K\= arlis Luksti\c n\v s,
Nat\= alija Somova,
Irina \v S\v cegu\c lnaja,
Anna Vanaga,
R\= usi\c n\v s Freivalds}
\institute{Institute of Mathematics and Computer Science,
 University of Latvia,\\ Rai\c na bulv\= aris 29, Riga, LV-1459, Latvia
}



\maketitle

\begin{abstract}  
We introduce a notion of ultrametric finite automata using $p$-adic numbers to describe random branching of the process of computation. These automata have properties similar to the properties of probabilistic automata but complexity of probabilistic automata and complexity of ultrametric automata can differ very much.
TODO
\end{abstract} 



\section{Introduction} 
%TODO: kaut ko sarakstiit / sakopeet no tiem 2 rakstiem 

\section{$p$-adic numbers}

Let $p$ be an arbitrary prime number. A $p$-adic number is a natural number between $0$ and $p-1$ (inclusive). A $p$-adic integer is by definition a sequence $(a_i)_{i \in N}$ of $p$-adic digits. We write this conventionally as
$$
\cdots a_i \cdots  a_2 a_1 a_0
$$
(that is, the $a_i$ are written from right to left).

If $n$ is a natural number, and
$$
n = \overline{a_{k-1} a_{k-2} \cdots  a_1 a_0}
$$
is its $p$-adic representation (in other words $n = \sum ^{k-1}_{i=0}  a_ip^i$ with each $a_i$ a $p$-adic digit) then we identify $n$ with the $p$-adic integer $(a_i)$ with $a_i = 0$ if $i \geq k$. This means that natural numbers are exactly the same thing as p-adic integers with a finite number of nonzero digits. Also note that $0$ is the $p$-adic integer all of whose digits are $0$, and that $1$ is the $p$-adic integer all of whose digits are $0$ except the right-most one (digit $0$) which is $1$.
If $\alpha  = (a_i)$ and $\beta = (b_i)$ are two $p$-adic integers, we will now define their sum. To that effect, we define by induction a sequence $(c_i)$ of $p$-adic digits and a sequence $(\epsilon _i)$ of elements of $\{0, 1\}$ (the "carries") as follows:


\begin{itemize}
  \item  $\epsilon _0 \mbox{ is } 0.$\\
  \item  $c_i \mbox{ is } a_i + b_i + \epsilon _i \mbox{ or } a_i + b_i + \epsilon _i - p \mbox{ according as which of these two is a }\\
p-\mbox{adic digit (in other words, is between 0 and p - 1). In the former case,}\\
\epsilon _i+1 = 0 \mbox{ and in the latter, } \epsilon _i+1 = 1.$
\end{itemize}


Under those circumstances, we let $\alpha + \beta = (c_i)$ and we call $\alpha + \beta $ the sum of $\alpha $ and $\beta $. Note that the rules described above are exactly the rules used for adding natural numbers in base $p$. In particular, if $\alpha $ and $\beta $ turn out to be natural numbers, then their sum as a $p$-adic integer is no different from their sum as a natural number. Addition is therefore associative and commutative.
Similarly, subtraction and multiplication of $p$-adic integers is done exactly the same as with natural numbers in base $p$.

Division of $p$-adics, however, cannot always be performed. For example, $\frac{1}{p}$ has no meaning as a $p$-adic integer - that is, the equation $p\alpha  = 1$  has no solution - since multiplying a $p$-adic integer by $p$ always gives a $p$-adic integer ending in $0$. There is nothing really surprising here: $\frac{1}{p}$ cannot be performed in the integers either.
However, what is mildly surprising is that division by $p$ is essentially the only division which cannot be performed in the p-adic integers.  This is one of the reasons why the notion of
$p$-adic integers is generalized and {\em $p$-adic numbers} are introduced. They are formal sequences of $p$-adic digits such that the sequence is infinite in the left-hand direction but finite in the right-hand direction. The notion of {\em $p$-adic dot} is introduced. The set of all $p$-adic numbers is denoted by 
$Q_p$.

For example, with $p = 7$ we show that the number $\alpha  = \cdots  333334$  is the number $\frac{1}{2}$ by adding it to it itself:
$$
\begin{tabular}{rrrrrrrrr|}
$\cdots$ &3 &3 &3 &3 &3 &4\\
$\cdots$ &3 &3 &3 &3 &3 &4\\
\hline
\hline
$\cdots$ &0 &0 &0 &0 &0 &1\\
\hline
\end{tabular}
$$
Thus, in the 7-adic integers, $\frac{1}{2}$ is an integer. And so are $\frac{1}{3}$ $(\cdots 44445)$, $\frac{1}{4}$ $(\cdots 1515152)$, $\frac{1}{5}$ $(\cdots 541254125413)$, $\frac{1}{6}$ $(\cdots 55556)$, $\frac{1}{8}$ $(\cdots 0606061)$, and so on.
But $\frac{1}{7}$, $\frac{1}{14}$ and so on, are not 7-adic integers. They are expressed as follows.
$$
\begin{tabular}{rrrrrrrr|}
$\cdots$ &0&0&0&0&.&1\\
\end{tabular}
$$
$$
\begin{tabular}{rrrrrrrrr|}
$\cdots$ &0&0&0&0&.&0&1\\
\end{tabular}
$$
It is important that $p$-adic numbers that are not $p$-adic integers and irrational real numbers are kind of incompatible. It is known that no $p$-adic number corresponds to $\sqrt{2}, \pi , e$ and there is a continuum of $p$-adic numbers not corresponding to any real number. Moreover, if $p_1 \neq p_2$ then $p_1$-adic and $p_2$-adic numbers also are incompatible.

However, $p$-adic numbers is not merely one of generalizations of rational numbers. They are very special because of the notion of {\em absolute value} of numbers.

If $X$ is a nonempty set, a distance, or {metric}, on $X$ is a function $d$ from pairs of elements $(x,y)$ of $X$ to the nonnegative real numbers such that
\begin{enumerate}
\item $d(x,y) = 0$ if and only if $x = y$,\\
\item $d(x,y) = d(y,x)$,\\
\item $d(x,y) \leq d(x,z) + d(z,y)$ for all $z \in X$.
\end{enumerate}

A set $X$ together with a metric $d$ is called a {\em metric space}. The same set $X$ can give rise to many different metric spaces.


{\em Norm} of an element $x \in X$ is the distance from $0$:
\begin{enumerate}
\item $\parallel x \parallel = 0$ if and only if $x = 0$,\\
\item $\parallel x.y \parallel = \parallel x \parallel . \parallel xy \parallel $,\\
\item $\parallel x+y \parallel  \leq \parallel x \parallel  + \parallel y \parallel $.
\end{enumerate}

We know one metric on $Q$  induced by the ordinary absolute value. However, there are other norms as well.

A norm  is called {\em ultrametric}  if the third requirement can be replaced by the stronger statement:
$\parallel x+y \parallel  \leq \max \{\parallel x \parallel  , \parallel y \parallel \}$.
Otherwise, the norm is called {\em Archimedean}.

\begin{definition}
Let $p \in \{2,3,5,7,11,13,\cdots \}$ be any prime number. For any nonzero integer $a$, let the $p$-adic ordinal (or valuation) of $a$, denoted $ord_p a$, be the highest power of $p$ which divides $a$, i.e., the greatest $m$ such that $a \equiv 0 (mod p^{m})$. For any rational number $x = a/b$, denote $ord_p x$ to be $ord_p a - ord_p b$. Additionally, $ord_p x = \infty $ if and only if $x = 0$.
\end{definition}

For example, the 7-adic valuation of 7 is 1. That of 14 is also 1, as are those of 21, 28, 35, 42 or 56. The 7-adic valuation of 49, on the other hand, is 2, as is that of 98. And the 7-adic valuation of 343 is 3. The 2-adic valuation of an integer is 0 iff it is odd, it is at least 1 iff it is even, at least 2 iff the integer is multiple by 4, and so on. The 7-adic valuation of $\frac{1}{7}$ is -1, and so are those of $\frac{3}{7}$, $\frac{1}{14}$, $\frac{5}{56}$. The 7-adic valuation of $\frac{1}{2}$ or $\frac{8}{3}$ is 0. The 7-adic valuation of $\frac{7}{3}$ or $\frac{14}{5}$ is 1. The 7-adic valuation of $\frac{48}{49}$ is -2.
We now define the $p$-adic absolute value (or $p$-norm) of a rational number $r$ to be $|r|_p = p^{-ord_p r}$. For example, $|p|_p = \frac{1}{p} , |1|_p = 1, |2p|_p = \frac{1}{p}$ ( if $p$ is odd), and $|\frac{1}{p^2}| = p^2$.

\begin{definition}
Let $p \in \{2,3,5,7,11,13,\cdots \}$ be any prime number. For arbitrary rational number $x$, its {\em $p$-norm} is:
\begin{center}
$|x|_p$ = \{
            \begin{tabular}{ccc}
            $\frac{1}{p^{ord_p x}}$, &  if  & $x \neq 0$;\\
            0,        &   if  & $x = 0$.\\
            \end{tabular}
\end{center}
\end{definition}

To make the equivalence of both definitions clearer, we say that the valuation of a $p$-adic number $(a_i)$ is the greatest $i_0$ such that $a_i = 0$ for all $i < i_0$. With this terminology, a $p$-adic integer is exactly a $p$-adic number with non negative valuation. And a small $p$-adic integer (one which ends in 0) is one whose valuation is (strictly) positive. It is not hard to check that this definition coincides with the aforementioned one for integers, hence for rationals. As for rationals, we define the $p$-adic absolute value and distance by $|\alpha |_p = p^{- ord_p \alpha }$. Note that the $p$-adic absolute value of a $p$-adic number is a real number (it is also a $p$-adic, and in fact a rational, but ought not to be considered as such).

Every rational number is a $p$-adic integer for some prime number $p$. The nature of irrational numbers is more complicated. For instance, $\sqrt{2}$ just does not exist as a $p$-adic number. On the other hand, there is a of continuum of $p$-adic numbers that are not real numbers. Also, there is a continuum of $p_1$-adic numbers not being $p_2$-adic numbers, and vice versa.

\section{Results}

%TODO: veel teoreemas

\begin{theorem}
For every $n>0$ there exists a language $L_n$ such that every deterministic finite automaton recognizing $L_n$ needs at least $2^n$ states, but there is an Ultrametric automaton recognizing $L_n$ with $2 n$ states.
\end{theorem}
\begin{proof}
%TODO
\qed
\end{proof}

%TODO: veel teoreemas

\end{document}
